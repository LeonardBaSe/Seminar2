\documentclass[conference]{IEEEtran}
\IEEEoverridecommandlockouts
%\documentclass[a4paper,12pt]{article}

%vorgegeben
\usepackage{balance}
\usepackage{graphicx}
\usepackage{booktabs}
\usepackage{relsize}
\usepackage{pgfplots}
\usepackage{tabularx}
\usepackage{gensymb}
\usepackage{caption}
\usepackage{listings}
\usepackage{babel}
\usepackage{siunitx}
\usepackage{url}
\usepackage{subcaption}
\usepackage{comment}
\usepackage{tikz}

%packages to eneble new font
\usepackage{fontspec}
\setmainfont{Myriad Pro}
%\setsansfont{Myriad Pro}
%\newfontfamily\computermodern{Computer Modern}
%\setsansfont{Latin Modern}

%no Idea why I have these
\usepackage[utf8]{inputenc}
\usepackage{amsmath}
\usepackage{hyperref}

%to enble captions vor the graphics
\usepackage{caption}

\usepackage{array} %enebels tabular

%neue pacages vom neuen Template
\usepackage{cite}
\usepackage{amssymb,amsfonts} %hier vorher: \usepackage{amsmath,amssymb,amsfonts}
\usepackage{algorithmic}
\usepackage{textcomp}
\usepackage{xcolor}
\usepackage{epstopdf}
\usepackage{multirow}
\usepackage{blindtext}
\pgfplotsset{compat=1.18}
\title{\textbf{Short Paper:} IoT-NDN: An IoT Architecture via Named Data
Netwoking (NDN)\\}
\author{Leonard Boetefuer}
\date{\today}


\begin{document}

\maketitle

\begin{abstract}
The Internet of Things (IoT) is gaining importance in everyday life, science, and industry. Although current IoT systems can be modeled in IP, future systems with over a billion devices will face great challenges. Data aggregation, naming scalability, and the handling of resource-restrained devices are problems that need to be solved. Named Data Networking (NDN)
addresses many issues by using named data. IoT based on NDN
(IoT-NDN) is used to solve many of the mentioned issues in this work.
\end{abstract}

\section{Introduction}
This short paper is based on \cite{b99}.
%The Internet of Things (IoT) is gaining importance in everyday life, science, and industry.
Currently, IoT is based on the Internet
Protocol (IP) which lacks scalability, robustness,
and efficiency. Mobility is not supported in location-based IP; protocols are needed for support.
Named Data Network (NDN) is data-focused and not location-focused; it allows devices to request data using unique, location-independent names. NDN offers scalability, lightweight configuration, and simplified communications, making NDN a solution for IoT systems \cite{b1}.

%\section{Related Work}
%braucht man das???? wahrscheinlich nicht

\section{Analysis of IoT and NDN}
This section will discuss the limitations of IoT devices and the challenges of the current Internet architecture.
\subsection{Challanges of the IoT}
1) \textbf{The connectivity of IoT devices:}
Currently, IoT devices use a server-client or host-to-host connection, neither is scalable enough for a billion devices.
In the host-host architecture, every host has to communicate with every other host, resulting in exponential resource consumption.
%100

2) \textbf{Technological Standards:}
The current standards are inadequate for network protocols, communication protocols, and data aggregation. They lead to inefficient caching and aggregation. 
In addition, mobility protocols are needed to mitigate the effect of a connection loss. 
%97

3) \textbf{Mobility:}
IoT systems that use mobile devices need to note, that devices are numerous and technologically diverse, and consumer reliance is increasing. 
% über 90

4) \textbf{Complexity and Integration Issues:}
IoT systems comprise many unique Application Programming Interfaces (APIs), protocols, and platforms. 
%APIs expeccially are not desined with the resource limitations of IoT devices in mind. 
The integration of technologies into the system is very complicated due to all the different combinations. This system should consider the resource limitation of its components.
%100

\subsection{NDN for the IoT}
1) \textbf{NDN Packet Length:}
Packages in NDN are not bound to a specific length, which allows the expansion of further protocols by adding or subtracting from the overhead.
IoT devices are limited in memory, resulting in small packages with a need for small overheads.
%100

2) \textbf{Caching in IoT/NDN:}
IoT devices have too little memory for efficient caching, resulting in increased unnecessary package flow and reduced data availability.
The solution is in-network caching, a feature of NDN.

3) \textbf{Data Aggregation in Wireless Networks:}
Interest packets are requests for data that are name-based. Data packets are their counterpart.
Multiple but very similar Data packets have to be requested separately (excluding the others). This results in a greater overhead and more unnecessary package flow. The new system should fix the request problem.
%98

4) \textbf{Naming Problems in Wireless Networks:}
NDN supports name-centric services, which facilitate access without knowing their location. 
 %Due to the size of the networks, there is still a need to automate naming conventions \cite{b17}. 
Data is addressed by names that should be kept short to minimize storage usage \cite{b18}. 

5) \textbf{Routing Scalability in NDN:}
The Pending Interest Table (PIT) tracks Interest packets until satisfaction. The Content Store (CS) is a cache that stores the Data packets. Routing scalability is important
to facilitate a large network \cite{b19}.

\section{Architecture of IoT-NDN System}

%die graphick muss an die richtige stelle.
\begin{figure}[h]
    \centering
    \includegraphics[width=0.3\textwidth]{IoT-NDN_System_architecture_and_its_components.png}\\
    \caption{IoT-NDN System architecture and its components}
    \label{fig:enter-label}
\end{figure}

The IoT-NDN has three main components that are responsible for packages, caching, strategies, and others.
The \textbf{naming} component, the \textbf{Management and control plane}, and the \textbf{dataplan} component. All three interact with each other, as seen in Fig. 1.

Devices in the IoT-NDN architecture have three tables: CS, PIT, and Forwarding Information Base (FIB).
%Was ist CS, PIT und FIB?

% The \textbf{naming} component, is made up of naming schemes and a structure for wireless networks.
% The \textbf{Management and control plane} is made up of Unicast Faces, Forwarding Services Intra Node Protocol, Controlled Flooding, Configuration, and Alias services. %aus Paper Zitiert. Vielleicht über graphic kürzen.
% The \textbf{dataplan} component is made up of the caching and forwarding strategies. 


%\includegraphics[width=0.3\textwidth]{IMPLEMENTED_AND_TESTED_PROTOCOLS_AND_ALGORITHMS_IN_IOT-NDN.png}\\


% {\tiny
% \begin{table}[h]
% \caption{Implemented and tested Protocols and Algorithms in IoT-NDN}
% \begin{tabular}{ | m{2,6cm} | m{1,5cm}| m{1,5cm} |  m{1,5cm} |} 

%   \hline
%   \textbf{IoT-NDN Protocols \& Algorithms}& \textbf{Implemented} & \textbf{Evaluation} & \textbf{References} \\ 
%   \hline
%   Naming and NDN Protocols & \checkmark & \checkmark & \cite{b4},\cite{b18} \\ 
%   \hline
%   Name-Centric Services & \checkmark & \checkmark & \cite{b18}\\ 
%   \hline
%   Efficient Caching Algorithm & \checkmark & \checkmark & \cite{b6},\cite{b7} \\ 
%   \hline
%   Adaptive Internet Forwarding & \checkmark & \checkmark & \cite{b6} \\ 
%   \hline
%   Control Flooding & \checkmark & \checkmark & \cite{b6} \\ 
%   \hline
%   Data Aggregation Protocol & \checkmark & \checkmark & \cite{b5} \\ 
%   \hline
%   APIs for IoT-NDN & \checkmark & \checkmark & \cite{b21},\cite{b22} \\ 
%   \hline

% \end{tabular}
% \end{table}
% }

\subsection{Naming}
%muss an die richtige stelle (unter naming)
\begin{figure}[h]
    \centering
    \includegraphics[width=0.3\textwidth]{Name_Structure_of_the_suggested_Approach.png}\\
    \caption{Name Structure of the suggested Approach}
    \label{fig:enter-label}
\end{figure}

In IoT-NDN the structure in which data are addressed is hierarchical, as shown in Fig. 2.
% The first component is a global domain name. 
% The second is optional and, if used, stores additional information, like the node ID.
% The third component contains the application and service name.
% The fourth component contains the command name, command ID, and additional parameters.
The marker component is the additional fifth component of the name, specific to IoT-NDN.
It contains information on the application, service, or device resources.
%The marker component can 
%was kann sie und wofür braucht man sie?

\subsection{Managment and Control Plane}
%ich kann 5 umsonst zitieren
1) \textbf{Aggregation:} Combining similar information into one package will reduce the memory stress and energy consumption of the CS \cite{b5}.
Three components achieve Data aggregation. The components are: Forwarding Service, Unicast Faces, and Intra Node Protocol (INP). They are explained in greater detail in \cite{b5}.

%erkläre Unicast Faces.#
%was ist radio layer (die sicherungsschicht?)
Unicast Face needs every package (in the radio layer) to include the source and destination address. 
When a device receives a package, it will build a connection to the sender of the package. 
%This enables devices to learn their neighbors. 
If a connection is lost, for this connection allocated resources will be released. 
%später noch ergänzen

2) \textbf{Controlled Flooding:}
%überarbeiet muss nochmal überarbeiet werden
IoT devices aren't reliable in power or connectivity; as a result, the FIB tables are usually insufficiently populated in advance by routing information. 
Controlled flooding is selected for its robustness. Expanding it with timer-based package suppression can reduce overhead. 
Packages won't be sent out while the timer is running, 
if the same package is received it will only be sent out once.

%muss ich 13 und 23 zitieren?
%IoT devices aren't reliable in power or connectivity. % 70%sicher
%As a result, the Forwarding Information Base (FIB) tables won't usually be populated in advance by routing information. %ziemlich eins zu eins übernommen
%Packages will transfer from one device to another and to mitigate flooding, controlled flooding is used.
%To reduce overhead and redundancy, devices will delay sending out packages. This time can be random or based on network topology.
%While a package is delayed and arrives  again at the same device, the package is not send out twice. 
%The path is selected
%MUSS NOCH PATH SELECTION MACHEN

3) \textbf{Name-Centric Services:}
The usage of a gateway enables IoT devices to be reachable from any internet device. 
The gateway allows wireless and wired connections, via protocol conversion. 
It provides all important configurations of names and  IoT-NDN devices. Services on the gateway can be developed as IoT-NDN applications, enabling communication to each other threw the IoT-NDN daemon and face. 
To reduce the workload on the IoT devices the gateway will use aliases (that are shorter) to reduce the size of the packages.
Names received from the internet will be mapped to names used in the IoT-NDN network.

\subsection{Data Plane}
1) \textbf{Strategy-In-Network Caching:} 
The CS differs from routers in IP by the ability to send cached packages more than once.
If a device receives a package request, it will first check the CS by checking for matching prefixes
because the same data will probably be requested many times in IoT-NDN networks.
The standard replacement strategy is LRU, but others can be implemented. %LFU und randome auch gut mit quelle 7 und 26
IoT-NDN devices have a special probabilistic CAching STrategy (pCASTING) 
that considers data freshness and the charge and storage of devices. 
This strategy is used %among other things
when a device receives a data package with a matching PIT.
% vieleicht mehr über pCasting, dann aber auch quelle 7 zitiern

2) \textbf{Strategy-Forwarding:}
The forwarding strategy component selects the forwarding path from the FIB to send the Interest packets. 
Remembering the number of unsatisfied interests, 
the forwarding strategy could be used to control the traffic. 
The forwarding component is the path of the Interest packet using data such as delay and throughput.
Forwarding steps on devices are supported by IoT-NDN and the forwarding strategy
allows the request of lost packages in a network by using its metrics. 
These metrics can be used to study the performance of every face. Finding missing packages
can also be achieved with the InterestLifeTime parameter. For
more information, see \cite{b6}
    
\section{Conclusion}
First, the paper highlights the problems of IoT if implemented in IP. 
%Then it examines cache mechanisms, data aggregation and naming in wireless networks.
NDN isn't built with resource-constrained devices in mind. This paper names the problems and solutions to integrate NDN into IoT.
The result Iot-NDN is a new type of network that includes communication and data access, based on names.
%97
    
\begin{thebibliography}{00}
\bibitem{b99}M. A. Heil, "IoT-NDN: An IoT Architecture via Named Data
Netwoking (NDN)" "in" Software Embedded System Department
Euroimmun AG, a PerkinElmer Company
\bibitem{b1} Z. Sheng, S. Yang, Y. Yu, A. V. Vasilakos, J. A. Mccann, and K. K.
Leung, “A survey on the ietf protocol suite for the internet of things:
standards, challenges, and opportunities,” IEEE Wireless Communications,
vol. 20, no. 6, pp. 91–98, December 2013.
% \bibitem{b4} Z. Ren, M. Hail, and H. Hellbruck, “Ccn-wsn - a lightweight, flexible
% content-centric networking protocol for wireless sensor networks,” in
% Intelligent Sensors, Sensor Networks and Information Processing, 2013.
\bibitem{b5} T. Teubler, M. A. M. Hail, and H. Hellbr¨uck, “Efficient Data Aggregation
with CCNx in Wireless Sensor Networks,” in 19th EUNICE Workshop
on Advances in Communication Networking (EUNICE 2013), Germany.
\bibitem{b6}M. A. Hail, M. Amadeo, A. Molinaro, and S. Fischer, “On the performance
of caching and forwarding in information centric networking for
the iot,” in 13th International Conference on Wired and Wireless Internet
% Communications, April 2015.
% \bibitem{b7}——, “Caching in named data networking for the wireless internet
% of things,” in Recent Advances in Internet of Things (RIoT), 2015
% International Conference on, April 2015, pp. 1–6.
% \bibitem{b17}V. Jacobson, D. K. Smetters, J. D. Thornton, M. F. Plass, N. H. Briggs,
% and R. L. Braynard, “Networking named content,” in Proceedings of the
% 5th International Conference on Emerging Networking Experiments and
% Technologies, ser. CoNEXT ’09. New York, NY, USA: ACM, 2009, pp.
1–12. [Online]. Available: http://doi.acm.org/10.1145/1658939.1658941
\bibitem{b18}T. Teubler, M. A. M. Hail, and H. Hellbr¨uck, “A solution for the
naming problem for name-centric services,” in Wired/Wireless Internet
Communications, A. Mellouk, S. Fowler, S. Hoceini, and B. Daachi,
Eds. Cham: Springer International Publishing, 2014, pp. 214–227.
    
    
    
    
\bibitem{b19}C. Yi, J. Abraham, A. Afanasyev, L. Wang, B. Zhang, and L. Zhang,
“On the role of routing in named data networking,” in Proceedings of
the 1st ACM Conference on Information-Centric Networking, ser. ACMICN
’14. New York, NY, USA: ACM, 2014, pp. 27–36.
    
    
    
    
% \bibitem{b21}S. Ebers, M. A. Hail, S. Fischer, and H. Hellbr¨uck, “Api for data
% dissemination protocols - evaluation with autocast,” in The Third World
% Congress on Nature and Biologically Inspired Computing (NaBIC 2011).
% Salamanca, Spain: IEEE, Oct. 2011, pp. 534–539.
% \bibitem{b22}M. A. Hail and S. Fischer, “Flexible API for IoT Services with Named
% Data Networking,” in 2016 IEEE International Conference on Emerging
% Technologies and Innovative Business Practices for the Transformation
% of Societies (IEEE EmergiTech 2016), Mauritius, Mauritius, Aug 2016.
\end{thebibliography}
\end{document}