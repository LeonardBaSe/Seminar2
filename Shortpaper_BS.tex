\documentclass[conference]{IEEEtran}
\IEEEoverridecommandlockouts
%\documentclass[a4paper,12pt]{article}

\usepackage{balance}
\usepackage{graphicx}
\usepackage{booktabs}
\usepackage{relsize}
\usepackage{pgfplots}
\usepackage{tabularx}
\usepackage{gensymb}
\usepackage{caption}
\usepackage{listings}
\usepackage{babel}
\usepackage{siunitx}
\usepackage{url}
\usepackage{subcaption}
\usepackage{comment}
\usepackage{tikz}

\usepackage{fontspec}
\setmainfont{Myriad Pro}
%\setsansfont{Myriad Pro}
%\newfontfamily\computermodern{Computer Modern}
%\setsansfont{Latin Modern}

\usepackage[utf8]{inputenc}
\usepackage{amsmath}
\usepackage{hyperref}

%neue pacages vom neuen Template
\usepackage{cite}
\usepackage{amssymb,amsfonts} %hier vorher: \usepackage{amsmath,amssymb,amsfonts}
\usepackage{algorithmic}
\usepackage{textcomp}
\usepackage{xcolor}
\usepackage{epstopdf}
\usepackage{multirow}
\usepackage{blindtext}
\pgfplotsset{compat=1.18}
\title{\textbf{Shortpaper:} IoT-NDN: An IoT Architecture via Named Data
Netwoking (NDN)\\}
\author{Leonard Boetefuer}
\date{\today}


\begin{document}

\maketitle

\begin{abstract}
als Leztes
\end{abstract}

\section{Introduction}
als vorleztes
\section{Related Work}
XYZTEST2
\section{Analysis of IoT and NDN}
This section will talk about the limitations of IoT devices and the challenges of the current Internet architecture.
\subsection{The connectivity of IoT devices:}
% The quick brown dog jumps over the lazy fox10.\\
% \sffamily
% The quick brown dog jumps over the lazy fox10.
Currently IoT devices use server-client or host-to-host connection to connect. 
In the server-client architecture every client has to communicate to the server and with a billion devices the server will be a massive bottleneck. In the host-host architecture, every host has to communicate to every other host. This results in exponential resource consumption.
The server-client and the host-to-host model both need IP addresses for every single device, which is not possible with a billion devices.
%check the ip adresses
%NDN solves the IP shortage and it is de-central helps with the bottleneck

\subsection{Technological Standards:}
The crucial standards are for the network protocols, the communication protocols, 
and the data aggregation. 
The challenge is that 
%überarbeiten 


\subsection{Mobility:}
The amount of mobile devices is rising and so are the challenges. 
The technologies of the mobile devices are divers and the IoT systems need to keep that in mind.

%Quelle 14 und 15

\subsection{Complexity and Integration Issues:}
IoT systems are composed of many different APIs (Application Programming Interfaces), 
protocols and platforms. 
%APIs expeccially are not desined with the resource limitations of IoT devices in mind. 
% maybe add that later
The integration of new technologies in the system is very complicated because of all the different combinations. 
The IoT system should consider the resource limitation of its components.

%AB HIER BEGINNT B

\subsection{NDN Packet Length:}
Packages in NDN are not bound to a specific length, this is helpful to expand further protocols by adding or subtracting from the overhead.
IoT devices will send only small packages; because of the limited resources in memory.
The overhead needs to be kept small, because the information proportion of small packages is way more influenced by a big overhead.
%to many packages will drain energy and put strain on the network.
%Quelle 16

\subsection{Caching in IoT/NDN:}
To keep the data up to date and reduce unnecessary package flow, we need to integrate caches in to the system. 
The problem is that the small devices don't have enough memory to keep an efficient cache. 
The solution is to use in-network caching, a feature of NDN.
%ganz kurz in-network caching erklären.
%Quelle 11
Even a small cache will dramatically increase the data availability [11].
%Quelle 2
\subsection{Data Aggregation in Wireless Networks:}
If a user's query requests for data that includes multiple packages, every single package has to be requested separately and excluding the others.
This results in a greater overhead and will result in a more unnecessary package flow. The new system should fix the request problem.

\subsection{Naming Problems in Wireless Networks:}
NDN supports on name-centric services [17], which facilitates access without knowing their location. There is still a need to automate naming convention, because of the size of the Networks. These Names should be kept short to minimize storage usage. 
%Quelle 18 (3 mal)


\subsection{Routing Scalability in NDN:}
In NDN routing is managed by names, instead of of usual number based systems. The scalability of routing is important to facilitate a large network.
% Da fehlt was mach das vielleicht später
% Quelle 19 und 20

\section{Architecture of IoT-NDN System}

\includegraphics[width=0.3\textwidth]{IoT-NDN_System_architecture_and_its_components.png}\\
IoT-NDN has three main components. The \textbf{naming} component is made up of naming schemes and structure for wireless networks.
The \textbf{Management and control plane} is made up of Unicast Faces; Forwardinng Services Intra Node Protocol, Controlled Flooding, Configuartion and Alias services. %aus Paper Zitiert. Vielleicht über graphic kürzen.
The \textbf{dataplan component} is made up of the caching and forwarding strategies.
Devices that use IoT-NDN


\section{Conclusion}


\includegraphics[width=0.3\textwidth]{IMPLEMENTED_AND_TESTED_PROTOCOLS_AND_ALGORITHMS_IN_IOT-NDN.png}\\
\includegraphics[width=0.3\textwidth]{Name_Structure_of_the_suggested_Approach.png}\\







%\subsection{Units}
% \begin{itemize}
% \item Related Work
% \item Analysis of IoT and NDN\\
%     A$)$ Challenges of IoT\\
%     \begin{enumerate}
%         \item The connection of IoT \\
%         the current paradigm are not usable for million or billions of devices.\\
%         The server-client architecture forces every device to communicate to the server first
%         and in the host-to-host architecture every device needs its own IP-adress.
        
%         \item Technology Standards\\
%         the current IoT technological standards have a problem with the aggregation of data.
%         the Data hase overhead to prevent data-aggregation, but the overhead consumes more energy and memory.
%         %nochmal angucken
        
%         \item Mobility\\
        
        
%         \item Complexity of Integration Issues
%     \end{enumerate}
%     B$)$ NDN for IoT\\
%     \begin{enumerate}
%         \item NDN Packet Lenght
%         \item Caching in IoT/NDN
%         \item Data Aggregation in Wireless Networks:
%         \item Naming Problems in Wirless Networks
%         \item Routing Scalability in NDN:
%     \end{enumerate}
% \item Architecture of IoT-NDN System\\
%     graphics and stuff\\
%     A$)$ Naming\\
%     B$)$ Management and Control Plane\\
%     \begin{itemize}
%         \item Aggregation
%         \item Controlled Flooding
%         \item Name-Centric Service
%     \end{itemize}
%     C$)$ Data PLane\\
%     \begin{itemize}
%         \item Strategy-In-Network Caching
%         \item Strategy-Forwarding
%     \end{itemize}
    
% \end{itemize}

% related Work

% short introduction\\

% A$)$ Challenges 1-5\\



\section{References}
XYZ
\end{document}